% Hi. This part is made by Pink. Please credit me here if you are to use this.
% https://github.com/TPColor/pinks-latex-boilerplate

\usepackage{amsmath, amssymb, amsthm} % basic packages
\usepackage{xparse}
\usepackage[dvipsnames,svgnames,table]{xcolor} % cool colors n other shit
\usepackage[most]{tcolorbox} % simple boxes
\usepackage[framemethod=TikZ]{mdframed} % framed boxes
\usetikzlibrary{shadows} % shadows no shit
\usepackage[margin=1in]{geometry} % margins
\usepackage{graphicx} % Required for inserting images
\usepackage{thmtools}
\usepackage{etoolbox} % for \pretocmd

\usepackage{mathdots} % for ascending diagonal dots
\usepackage{wasysym} % for lightning (contradiction)
\newcommand{\whenever}{\mathrel{\reflectbox{$\implies$}}} % converse
\usepackage{upgreek} % need uptau

% Hi. This part is made by Pink. Please credit me here if you are to use this.
% https://github.com/TPColor/pinks-latex-boilerplate

% define colors
\definecolor{wisteria}{HTML}{C9A0DC}


% definition styles

% black
\mdfdefinestyle{mddefbox}{
    %roundcorner=10pt,
    linewidth=1pt,
    skipabove=12pt,
    skipbelow=2pt,
    innertopmargin=9pt,
    innerbottommargin=9pt,
    linecolor=black,
    nobreak=true,
    backgroundcolor=black!5,
    shadow=true,
    shadowsize=6pt,
    shadowcolor=black!30,
}
\declaretheoremstyle[
    headfont=\sffamily\bfseries\color{black},
    mdframed={style=mddefbox},
    headpunct={\\[3pt]},
    postheadspace={0pt}
]{defstyle}

% teal
\mdfdefinestyle{mdpropbox}{
    roundcorner=10pt,
    linewidth=1pt,
    skipabove=2pt,
    skipbelow=2pt,
    innertopmargin=9pt,
    innerbottommargin=9pt,
    linecolor=Teal,
    nobreak=true,
    backgroundcolor=Teal!5,
    shadow=true,
    shadowsize=6pt,
    shadowcolor=black!30
}
\declaretheoremstyle[
    headfont=\sffamily\bfseries\color{Teal},
    mdframed={style=mdpropbox},
    headpunct={\\[3pt]},
    postheadspace={0pt}
]{propstyle}

% wisteria
\mdfdefinestyle{mdlemstyle}{
    roundcorner=10pt,
    linewidth=1.5pt,
    skipabove=2pt,
    skipbelow=2pt,
    innertopmargin=9pt,
    innerbottommargin=9pt,
    linecolor=wisteria,
    nobreak=true,
    backgroundcolor=wisteria!15,
    shadow=true,
    shadowsize=6pt,
    shadowcolor=black!30
}
\declaretheoremstyle[
    headfont=\sffamily\bfseries\color{wisteria!95!black},
    mdframed={style=mdlemstyle},
    headpunct={\\[3pt]},
    postheadspace={0pt}
]{lemstyle}

% magenta
\mdfdefinestyle{mdthmbox}{
    roundcorner=10pt,
    linewidth=2pt,
    skipabove=12pt,
    skipbelow=2pt,
    innertopmargin=9pt,
    innerbottommargin=9pt,
    linecolor=magenta,
    nobreak=true,
    backgroundcolor=magenta!10,
    shadow=true,
    shadowsize=6pt,
    shadowcolor=black!30,
}
\declaretheoremstyle[
    headfont=\sffamily\bfseries\color{magenta},
    mdframed={style=mdthmbox},
    headpunct={\\[3pt]},
    postheadspace={0pt}
]{thmstyle}

% orange
\mdfdefinestyle{mdexbox}{
    %roundcorner=10pt,
    linewidth=1pt,
    skipabove=12pt,
    skipbelow=2pt,
    innertopmargin=9pt,
    innerbottommargin=9pt,
    linecolor=RawSienna,
    nobreak=true,
    backgroundcolor=Salmon!5,
    shadow=true,
    shadowsize=6pt,
    shadowcolor=black!30,
}
\declaretheoremstyle[
    headfont=\sffamily\bfseries\color{RawSienna},
    mdframed={style=mdexbox},
    headpunct={\\[3pt]},
    postheadspace={0pt}
]{exstyle}

% curve lazy (wisteria aka pastel purple)
\mdfdefinestyle{mdcurvelazybox}{
    roundcorner=10pt,
    linewidth=1pt,
    skipabove=12pt,
    skipbelow=2pt,
    innertopmargin=9pt,
    innerbottommargin=9pt,
    linecolor=LimeGreen,
    nobreak=true,
    backgroundcolor=LimeGreen!5,
    shadow=true,
    shadowsize=6pt,
    shadowcolor=black!30,
}
\declaretheoremstyle[
    headfont=\sffamily\bfseries\color{LimeGreen},
    mdframed={style=mdcurvelazybox},
    headpunct={\\[3pt]},
    postheadspace={0pt}
]{curvelazystyle}

% brick lazy (wisteria aka pastel purple)
\mdfdefinestyle{mdbricklazybox}{
    %roundcorner=10pt,
    linewidth=1pt,
    skipabove=12pt,
    skipbelow=2pt,
    innertopmargin=9pt,
    innerbottommargin=9pt,
    linecolor=LimeGreen,
    nobreak=true,
    backgroundcolor=LimeGreen!5,
    shadow=true,
    shadowsize=6pt,
    shadowcolor=black!30,
}
\declaretheoremstyle[
    headfont=\sffamily\bfseries\color{LimeGreen},
    mdframed={style=mdbricklazybox},
    headpunct={\\[3pt]},
    postheadspace={0pt}
]{bricklazystyle}

% fuchsia
\mdfdefinestyle{mdprobbox}{
    roundcorner=10pt,
    linewidth=1.5pt,
    skipabove=12pt,
    skipbelow=2pt,
    innertopmargin=9pt,
    innerbottommargin=9pt,
    linecolor=Fuchsia!50!Red!90!black,
    nobreak=true,
    backgroundcolor=PeachPuff!25,
    shadow=true,
    shadowsize=6pt,
    shadowcolor=black!30,
    frametitleaboveskip=8pt,
    frametitlebelowskip=8pt,
    frametitlebackgroundcolor=Fuchsia!50!Red!90!black,
    frametitlefont=\bfseries\sffamily\color{PeachPuff!25},
    frametitlerule=true
}


% environment styles for basic boxes
\tcbset{
  subproblem box/.style={
    enhanced,
    boxrule=0pt,
    frame hidden,
    sharp corners,
    colback=PeachPuff!25,
    borderline west={3pt}{0pt}{Fuchsia!50!Red!90!black},
    before skip=6pt,
    after skip=6pt,
    left=10pt,
    right=10pt,
    breakable
    },
%
  subsubproblem box/.style={
    enhanced,
    boxrule=0pt,
    frame hidden,
    sharp corners,
    colback=PeachPuff!25,
    % borderline west={3pt}{0pt}{Fuchsia!50!Red!90!black},
    before skip=6pt,
    after skip=6pt,
    left=10pt,
    right=10pt,
    breakable
    },
%
  observation box/.style={
    enhanced,
    boxrule=0pt,
    frame hidden,
    sharp corners,
    colback=wisteria!15,
    coltext=wisteria!95!black,
    borderline west={3pt}{0pt}{wisteria},
    before skip=6pt,
    after skip=6pt,
    left=10pt,
    right=10pt,
    breakable
    },
%
  claim box/.style={
    enhanced,
    boxrule=0pt,
    frame hidden,
    sharp corners,
    colback=pink!70!magenta!15,
    coltext=pink!60!magenta,
    borderline west={3pt}{0pt}{pink!70!magenta},
    before skip=6pt,
    after skip=6pt,
    left=10pt,
    right=10pt,
    breakable
    },
%
  remark box/.style={
    enhanced,
    boxrule=0pt,
    frame hidden,
    sharp corners,
    colback=DeepSkyBlue!5,
    coltext=black,
    borderline west={3pt}{0pt}{DeepSkyBlue},
    before skip=6pt,
    after skip=6pt,
    left=10pt,
    right=10pt,
    breakable
    },
%
  corollary box/.style={
    enhanced,
    boxrule=0pt,
    frame hidden,
    sharp corners,
    colback=magenta!10,
    coltext=black,
    borderline west={3pt}{0pt}{magenta},
    before skip=6pt,
    after skip=6pt,
    left=10pt,
    right=10pt,
    breakable
    },
%
  remark box/.style={
    enhanced,
    boxrule=0pt,
    frame hidden,
    sharp corners,
    colback=DeepSkyBlue!5,
    coltext=black,
    borderline west={3pt}{0pt}{DeepSkyBlue},
    before skip=6pt,
    after skip=6pt,
    left=10pt,
    right=10pt,
    breakable
    },
%
  domino box/.style={
    enhanced,
    boxrule=0pt,
    frame hidden,
    sharp corners,
    colback=black!5,
    coltext=black,
    borderline west={3pt}{0pt}{black},
    before skip=6pt,
    after skip=6pt,
    left=10pt,
    right=10pt,
    breakable
    },
}


% theorem styles for basic boxes

% corollary
\declaretheoremstyle[
	headfont=\bfseries\sffamily\color{magenta},
	bodyfont=\normalfont,
	headpunct={\\[3pt]},
]{corostyle}

% remark
\declaretheoremstyle[
	headfont=\bfseries\sffamily\color{DeepSkyBlue},
	bodyfont=\normalfont,
	headpunct={\\[3pt]},
]{remstyle}

% domino
\declaretheoremstyle[
	headfont=\bfseries\sffamily\color{black},
	bodyfont=\normalfont,
	headpunct={\\[3pt]},
]{dominostyle} 

% fancy "definition" environments
\declaretheorem[style=defstyle,name=Definition,numberwithin=section]{definition}
\declaretheorem[style=defstyle,name=Definition,numbered=no]{definition*} % non-numbered definitions
\declaretheorem[style=propstyle,name=Proposition,sibling=definition]{proposition}
\declaretheorem[style=lemstyle,name=Lemma,sibling=definition]{lemma}
\declaretheorem[style=thmstyle,name=Theorem,sibling=definition]{theorem}
\declaretheorem[style=exstyle,name=Example,sibling=definition]{example}
\declaretheorem[style=curvelazystyle,name=Curve Placeholder,sibling=definition]{curve}
\declaretheorem[style=bricklazystyle,name=Brick Placeholder,sibling=definition]{brick}

% basic "remark" environments
\declaretheorem[style=remstyle,name=Remark,sibling=definition]{remark}
\tcolorboxenvironment{remark}{remark box}
\declaretheorem[style=remstyle,name=Fact,sibling=definition]{fact}
\tcolorboxenvironment{fact}{remark box}
\declaretheorem[style=corostyle,name=Corollary to Theorem,sibling=definition]{corollary}
\tcolorboxenvironment{corollary}{corollary box}
\pretocmd{\corollary}{\addtocounter{corollary}{-1}}{}{} % decrement counter of corollary specifically
\declaretheorem[style=corostyle,name=Corollary to Proposition,sibling=definition]{corollaryprop}
\tcolorboxenvironment{corollaryprop}{corollary box}
\pretocmd{\corollaryprop}{\addtocounter{corollaryprop}{-1}}{}{} % decrement counter of corollary specifically
\declaretheorem[style=dominostyle,name=Domino,sibling=definition]{domino}
\tcolorboxenvironment{domino}{domino box}

% problem environments
\newenvironment{problem}[1]{
    \begin{mdframed}[style=mdprobbox, frametitle={#1}]
}{
    \end{mdframed}
}
\newtcolorbox{subproblem}[1][]{subproblem box, before upper={\textbf{#1}}}
\newtcolorbox{subsubproblem}[1][]{subsubproblem box, before upper={\textbf{#1}}}

% solution environments
\newcommand{\magentasquare}{\textcolor{magenta}{\(\Box\)}} % magenta qed
\newcommand{\moneyqed}{\hfill\includegraphics[height=2ex]{media/money mouth emoji.png}} % money mouth qed
\newenvironment{solution}[1][Solution]{
  \par\noindent\textit{#1.} \ignorespaces
}{
  \moneyqed\par 
}

% claim environments
\newtcolorbox{observation}[1]{
    observation box,
    before upper={
        \sffamily\bfseries#1\textbf{ --- }\color{black}\normalfont
    }
}
\newtcolorbox{claim}[1]{
    claim box,
    before upper={
        \sffamily\bfseries#1\textbf{ --- }\color{black}\normalfont
    }
}